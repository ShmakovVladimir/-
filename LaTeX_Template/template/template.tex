\documentclass[a4paper, 12pt]{extarticle}
\usepackage[dvipsnames]{xcolor}
\usepackage[top=70pt,bottom=70pt,left=48pt,right=46pt]{geometry}
\definecolor{header}{RGB}{252, 171, 16}
\definecolor{defenition}{RGB}{248, 51, 60}
\definecolor{main_title}{RGB}{43, 158, 179}
\definecolor{sub_header}{RGB}{68, 175, 105}
\usepackage[english, russian]{babel}
\usepackage[utf8]{inputenc}
\usepackage{amsmath}
\usepackage[most]{tcolorbox}
\usepackage{listings}
\usepackage{graphicx}
\usepackage{amsmath}
\usepackage{lettrine}
\title{\textcolor{main_title}{Очень интересная работа}}
\author{Шмаков Владимир Евгеньевич - ФФКЭ гр. Б04-103}


\newtcolorbox{fequation}[1][]{ams equation*,size=small,#1}








\begin{document}
\maketitle



\section*{\textcolor{header}{Цель работы}}
\section*{\textcolor{header}{Теоретические сведения}}

\lettrine{\textcolor{defenition}{П}}{\textcolor{defenition}{реобразование Фурье}} — операция, сопоставляющая одной функции вещественной переменной другую (вообще говоря, комплекснозначную) функцию вещественной переменной. Эта новая функция описывает коэффициенты («амплитуды») при разложении исходной функции на элементарные составляющие — гармонические колебания с разными частотами.

\begin{fequation}
    \hat{f}(\omega) = \sqrt{\frac{1}{2 \pi}} \int_{-\infty}^{\infty} f(t) e^{-i\omega t} d t
\end{fequation}


\section*{\textcolor{header}{Методика}}

\subsection*{\textcolor{sub_header}{Оборудование}}


\subsection*{\textcolor{sub_header}{Экспериментальная установка}}

\section*{\textcolor{header}{Обработка экспериментальных данных}}



\section*{\textcolor{header}{Вывод}}






\end{document}
